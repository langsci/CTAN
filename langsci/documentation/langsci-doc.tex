\documentclass[%
modfonts,
output=guidelines,
guidelines]{langscibook}
\author{Sebastian Nordhoff} 
\title{The \texttt{langscibook} {\LaTeX} class}
\subtitle{}
\renewcommand{\lsSeries}{guidelines}
\renewcommand{\lsSeriesNumber}{~}
\renewcommand{\lsAdvertisement}{}
% \BackTitle{} 
% \BackBody{}
% \dedication{}
% \typesetter{}
% \proofreader{}
% \renewcommand{\lsISBNdigital}{000-0-000000-00-0}
% \renewcommand{\lsISBNhardcover}{000-0-000000-00-0}
% \renewcommand{\lsISBNsoftcover}{000-0-000000-00-0}
% \renewcommand{\lsISBNsoftcoverus}{000-0-000000-00-0}
% \BookDOI{}
% \renewcommand{\lsURL}{http://langsci-press.org/catalog/book/000} 

\bibliography{localbibliography}

\newcommand{\option}[3]{\subsection{\texttt{#1}}{#2}\\Default value: {\texttt{#3}}}
\begin{document}
\maketitle

\frontmatter
\currentpdfbookmark{Contents}{name} % adds a PDF bookmark
\tableofcontents
% \addchap{Preface}
\begin{refsection}

%content goes here

\printbibliography[heading=subbibliography]
\end{refsection}


% \addchap{Acknowledgments}
\begin{refsection}

%content goes here

\printbibliography[heading=subbibliography]
\end{refsection}


% \addchap{Abbreviations}
% \addchap{Abbreviations and symbols}

\begin{multicols}{2} 


\begin{tabular}{lp{4.5cm}} 
...        &    ... \\
...        &    ... \\
...        &    ... \\
...        &    ... \\
\end{tabular}
%
\begin{tabular}{lp{4.5cm}} 
...        &    ... \\
...        &    ... \\
...        &    ... \\
...        &    ... \\
\end{tabular}

 
\end{multicols}  

\mainmatter       
\chapter{Introduction}
This document describes the {\LaTeX} class langscibook.cls. This class is used for typesetting books with Language Science Press. Language Science Press publish open access monographs and edited volumes in linguistics. 


\chapter{The class}
The class is loaded via \verb+\documentclass{langscibook}+. The standard use case is the creation of a book. If used without an \texttt{output=} option, the option \texttt{output=book} is loaded. The class can also be used to create dust jackets for books with different measurements via the output option values 
\texttt{covercreatespace} (for \url{http://www.createspace.com}) and
\texttt{coverbod} (for \url{http://www.bod.de}).

The class provides for the following:
\begin{itemize}
 \item front cover
 \item frontispiece 
 \item inside title 
 \item colophon 
 \item table of contents
 \item chapters 
 \item list of references (per chapter or global)
 \item name index, subject index, language index
 \item advertisement 
 \item back title 
 \item ISBN
 \item DOI (per book and per chapter)
\end{itemize}

When used to create dust jackets, the class will produce a large pdf with a collation of backcover, spine and frontcover, from left to right, surrounded by bleed.


\chapter{Requirements}
\section{Fonts}
The class uses the fonts Libertine, Arimo and Dejavu. These have to be present on the system. 

\section{Styles}
A couple of additional style files come with the class. Documents using this class will compile without them, but in order to typeset books for Language Science Press, these packages should be loaded as well. These packages are

\begin{itemize}
 \item \texttt{langsci-basic.sty}: Commands used in all Language Science Press books
 \item \texttt{langsci-optional.sty}: Further commands useful for some typical books
 \item \texttt{langsci-gb4e.sty}: Package for linguistic examples
 \item \texttt{langsci-cgloss.sty}: Helper package for \texttt{langsci-gb4e}
 \item \texttt{langsci-forest-setup.sty}: Set up the forest package for linguistic trees
%  \item \texttt{langsci-lgr.sty}: 
 \item \texttt{langsci-tbls.sty}: Package for the textbook series Textbooks in Language Sciences
 \item \texttt{langsci-bidi.sty}: Package for typesetting right-to-left text
\end{itemize}

\section{Colors and series definitions}
The metadata for Language Science Press book series are stored in the file \texttt{series.def}. This file loads the file \texttt{colors.def} to set the colors for the cover and title font. Both files should be safe to use for all books of well-established series. For new series, the ISSN field of \texttt{series.def} might have to be updated. 

\chapter{Creating a book}
The easiest way to start a book is to download the skeletons from \url{http://langsci-press.org/templatesAndTools}. Sample documents for monographs, edited volumes and chapters in edited volumes are provided in the folder \texttt{samples/} as well. 

\section{Creating a monograph}
Load the class. Make sure the following metadata values are set, either as options or with the metadata commands listed in \sectref{sec:metadatacommands}:

\begin{itemize}
 \item author 
 \item title
 \item series 
 \item series number
 \item all ISBNs
 \item BookDOI
 \item URL
 \item blurb (back body)
\end{itemize}

Provide the following additional information if necessary:
\begin{itemize}
 \item subtitle
 \item backtitle
 \item spine title 
 \item spine author
 \item dedication 
 \item list of proofreaders 
 \item list of typesetters
 \item list of illustrators 
 \item license other than CC-BY
\end{itemize}

 
\section{Creating an edited volume}
Proceed as for a monograph, but use the option \texttt{collection}. Use \verb+\author+ for the editor(s). You may want to use the further options \texttt{collectionchapter} and \texttt{collectiontoclong}. Chapters are included via \verb+\includepaper{}+. Set the bibliography resource for all papers in the main file. 


\section{Creating a paper for an edited volume}
Use the output option \texttt{paper}. Provide the following metadata in the preamble:
\begin{itemize}
 \item \verb+\author+ with (\verb+\affiliation+)
 \item \verb+\title+
 \item \verb+\abstract+ (optional) 
\end{itemize}

You might want to redefine \verb+\rohead+ for chapters with very long titles. 

 
\chapter{Creating a dust jacket}
Use the option \texttt{coverbod} or \texttt{covercreatespace}. Set the lengths  \verb+\bodspine+ and \verb+\csspine+. For \texttt{covercreatespace} set the further output option \texttt{coverus} if desired. Run {\XeLaTeX} twice. Upload to the respective websites to see whether the lengths were chosen correctly. 

\chapter{Options}
The class has flag options to toggle certain behaviour and string options to set metadata values. 
The metadata values can all also be set via the metadata commands listed in \sectref{sec:metadatacommands}.

\section{Flag options}
% \option{blackandwhite}
% Remove all colors}
% {false
% \option{smallfont}
% use 10pt as fontsize
% {false  
\option{draftmode}{Switch to draft mode (adds: draft stamp, indication of overlong lines, date)}{false} 
\option{openreview}{Switch to open review mode}{false}
% \option{noindex}
% remove index}
% {false  
\option{nonflat}{Use elaborate directory structure. If set to true, fonts, packages etc will be loaded from subfolders. If set to false, they will be loaded from the working directory}{false}  
\option{modfonts}{Use modified fonts provided by Language Science Press}{false}  
\option{showindex}{Show indexed terms in margin}{false}
\option{biblatex}{Use Bib\LaTeX}{true}
\option{bibtex}{Use Bibtex}{false}  
\option{newtxmath}{Switch math fonts to newtxmath}{false}
\option{collection}{Make the book an edited volume}{false}  
\option{collectionchapter}{Add numeric chapter prefix to each contribution.}{false}
\option{collectiontoclong}{More detailed table of content in edited volumes}{false}
\option{coverus}{Use \texttt{isbnsoftcoverus} instead of \texttt{isbnsoftcover} if \texttt{output=covercreatespace}}{false}

\section{String options}
\option{number}{Number of the book within the series}{"??"}
\option{issn}{ISSN of the series this book will appear in}{"??"}
\option{isbndigital}{The ISBN of the digital release}{"000-0-000000-00-0"}
\option{isbnsoftcover}{The ISBN of the soft cover release}{"000-0-000000-00-0"}
\option{isbnsoftcoverus}{The ISBN of the US version of soft cover release (used for distribution to US academic institutions)}{"000-0-000000-00-0"}
\option{isbnhardcover}{The ISBN of the hard cover release}{"000-0-000000-00-0"}
\option{url}{The URL of the book }{"http://langsci-press.org/catalog"}
\option{series}{The series code (see \texttt{series.def} for a list of acronyms)}{"eotms"}
\option{output}{Different output formats}{book}
\begin{itemize}
\item \texttt{book}: The book with frontcover and backcover 
\item \texttt{inprep}: Mark manuscript as in preparation
\item \texttt{paper}: Contribution to edited volume
\item \texttt{guidelines}: Choose color grey; condense frontmatter
\item \texttt{coverbod}: Create a two-page wide cover  (back-spine-front) for use with BoD
\item \texttt{covercreatespace}: Create a two-page wide cover  (back-spine-front) for use with CreateSpace 
\end{itemize}
\option{copyright}{The license chosen}{"CC-BY"}
\option{biblatexbackend}{The backend of BibLaTeX}{"bibtex"}
Alternative: \texttt{biber}
% \backmatter
% \phantomsection%this allows hyperlink in ToC to work
% \printbibliography[heading=references] 
% \cleardoublepage
% \phantomsection 
% \addcontentsline{toc}{chapter}{Index} 
% \addcontentsline{toc}{section}{Name index}
% \ohead{Name index} 
% \printindex 
% \cleardoublepage
% \phantomsection 
% \addcontentsline{toc}{section}{Language index}
% \ohead{Language index} 
% \printindex[lan] 
% \cleardoublepage  
% \phantomsection 
% \addcontentsline{toc}{section}{Subject index}
% \ohead{Subject index} 
% \printindex[sbj]
% \ohead{} 




\section{Metadata commands}\label{sec:metadatacommands}
The following commands are used to indicated metadata. These commands can override the options passed to the class.

\option{$\backslash$BackBody}{Text to be printed on the back cover}{Europan lingues es membres del sam familie. Lor separat existentie es un
myth. Por scientie, musica, sport etc, litot Europa usa li sam vocabular. Li lingues differe solmen in li grammatica, li pronunciation e li plu commun vocabules. Omnicos directe al desirabilite de un nov lingua franca: On refusa continuar payar custosi traductores.}
\option{$\backslash$BackTitle}{Title on back cover}{\textrm{same as on front cover}}
\option{$\backslash$BookDOI}{DOI for the book}{??}
\option{$\backslash$ChapterDOI}{DOI for a chapter}{??}
\option{$\backslash$dedication}{A dedication}{\textrm{void}}
\option{$\backslash$ISBNdigital}{ISBN for the digital version}{000-0-000000-00-0}
\option{$\backslash$ISBNhardcover}{ISBN for the hardcover version}{000-0-000000-00-0}
\option{$\backslash$ISBNsoftcover}{ISBN for the softcover version}{000-0-000000-00-0}
\option{$\backslash$ISBNsoftcoverus}{ISBN for the softcover version (US distribution)}{000-0-000000-00-0}
\option{$\backslash$Series}{The series this book appears in. Use the acronyms found in \texttt{series.def}}{eotms} 
\option{$\backslash$SeriesNumber}{Number of the book in the series}{??} 
\option{$\backslash$URL}{The URL where this title can be downloaded}{http://langsci-press.org/catalog} 


\section{Overrides}
The following commands allow to override further default settings:

\option{$\backslash$SpineAuthor}{Custom author on spine. Useful for long lists of authors/editors}{\textrm{same as on cover}} 
\option{$\backslash$SpineTitle}{Custom title on spine. Useful for long titles}{\textrm{same as on cover}} 

\section{Command redefinitions}
Redefine the following commands to further alter the appearance:

\option{$\backslash$lsAdvertisement}{specify the advertisement on the last page of the book}{$\backslash$include\{$\backslash$logopath didyoulikethisbook\}}
\option{$\backslash$lsBackBodyFont}{Font used for text on back cover}{\textrm{main font}} 
\option{$\backslash$lsBackTitleFont}{Font used for title on back cover}{$\backslash$sffamily$\backslash$addfontfeatures{Scale=MatchUppercase}\\$\backslash$fontsize\{25pt\}\{10mm\}$\backslash$selectfont}
\option{$\backslash$lsCopyright}{Choice of licence}{CC-BY}
\option{$\backslash$lsCoverAuthorFont}{Font for author/editor on cover}{$\backslash$fontsize\{25pt\}\{12.5mm\}$\backslash$selectfont}
\option{$\backslash$lsCoverSubTitleFont}{Font for the subtitle}{$\backslash$sffamily$\backslash$addfontfeatures\{Scale=MatchUppercase\}\\
$\backslash$fontsize\{25pt\}\{10mm\}$\backslash$selectfont}
% \option{$\backslash$lsCoverTitleFont 
\option{$\backslash$lsEditorPrefix}{How editors are indicated on title page. Useful for localization.}{{$\backslash$LARGE Edited by}\\} 
\option{$\backslash$lsFontsize}{size of main font}{11pt}
\option{$\backslash$lsImpressumCitationText}{custom citation text, useful for including notes etc.}{\textrm{computed from author and title fields}} 
\option{$\backslash$lsISSN}{ISSN of the series}{??}
\option{$\backslash$lsLanguageIndexTitle}{The title of the language index. Change this for non-English works}{Language index} 
\option{$\backslash$lsNameIndexTitle}{The title of the name index. Change this for non-English works}{Name index} 
\option{$\backslash$lsSpineAuthorFont}{Font for the author on the spine}{$\backslash$fontsize\{16pt\}\{14pt\}$\backslash$selectfont}
\option{$\backslash$lsSpineTitleFont}{Font for the title on the spine}{$\backslash$sffamily$\backslash$fontsize\{18pt\}\{14pt\}$\backslash$selectfont}
\option{$\backslash$lsSubjectIndexTitle}{The title of the subject index. Change this for non-English works}{Subject index} 
\section{Additions}
\option{$\backslash$lsAdditionalFontsImprint}{Add additional fonts to the list of fonts in the colophon}{\textrm{void}}


\end{document}